\documentclass[aspectratio=169]{beamer}

\usetheme{Janelia}

\title[imglyb]{imglyb\\[0.3em]Bridging The Chasm Between ImageJ and NumPy}
\subtitle{imglyb}
\author{Philipp Hanslovsky}
\date{July 10, 2019}

\setcounter{showSlideNumbers}{1}

\begin{document}

\setcounter{showProgressBar}{0}
\setcounter{showSlideNumbers}{0}

\maketitle

\section{Introduction}
\begin{frame}
    \frametitle{}
    Tell story: About five 1/2 years ago I moved from Heidelberg, Germany all the way across the
    Atlantic to Ashburn, VA, just outside Washington DC. After arrival I learned that I would have a
    much bigger journey still ahead of me. I first started programming seriously during my Master's
    when I worked on cell tracking in developing embryos in the Hamprecht lab at Heidelberg
    University. The lab was and is still creating high-performance software in C++ and with Python
    wrappers for easy interaction. Naturally, I was exposed to C++ and Python, in particular the
    NumPy and SciPy frameworks as depicted here as Mount SciPy. It took a lot learning and effort
    but eventually I managed to climb Mount SciPy and was comfortable writing software in
    the SciPy framework. The wealth of concise array arithmetic and scientific libraries help
    developers create efficient software with a minimal number of lines of very readable code.

    After my move to the US, I realized that not everybody is doing in Image Processing in C++ or
    Python: For the first time in my life, I had to program in the Java programming
    language. Contrary to popular misconception, image processing can be very efficient in Java, but
    it is very different than in Python. I had to climb another mountain, Mount ImgLib2, and many
    times I stumbled and fell down because previously learned patterns in the SciPy environment were
    not valid anymore. After a long but successful ascent, I finally understood and learned to
    appreciate the design: Pixel access is virtualized and independent from how the data is
    provided: For example, as Java arrays, chunked images, or as a function of the
    coordinate. Sometimes I find myself on top of Mount ImgLib2 happily coding away and then I look
    over to Mount SciPy and I can see all the neat SciPy libraries and Python features like operator
    overloading for concise notation of array operations. Mount SciPy seems so close but there is
    this huge chasm in between that separates those two peaks. I created imglyb to bridge this
    chasm for shared memory access of NumPy and ImgLIb2 data structures.

    This talk will consist of three parts:
    \begin{enumerate}
          \item Why is it so hard
          \item Way out: What did I do and how can you install it?
          \item Examples
          \item Known issues
    \end{enumerate}
\end{frame}

\section{imglyb}
\begin{frame}
    \frametitle{Why is it hard?}
    \begin{itemize}
          \item native vs JVM
          \item difference between c array/pointer and Java array
    \end{itemize}
\end{frame}

\begin{frame}
    \frametitle{Ways out?}
    \begin{itemize}
          \item Py4j: Independent Java and Python processes communicate~--- no shared memory
          \item Start Python process within Java and expose Python classes through JNI: Cool but
        cannot use Python syntax like operator overloading
          \item JyNI Re-implement Python in Java~(Jython) and make CPython libraries accessible through JNI:
        Jython only 2.7
          \item Start a JVM in Python process through JNI and make native memory accessible in ImgLib2 
    \end{itemize}
\end{frame}


\begin{frame}
    \frametitle{Way out!}
    \begin{itemize}
          \item Start JVM in Python process
          \item Make Java methods and classes available in Python
          \item Create ImgLib2 data structures that can read from and write into native memory
          \item Create ImgLib2 data structures that understand NumPy meta data such as strides or
        offset into native memory
          \item Create a Python library that wraps numpy.ndarray into ImgLib2 data structures with
        shared memory access
    \end{itemize}
\end{frame}

\begin{frame}
    \frametitle{Installation \& Usage}
    conda install -c conda-forge imglyb
    pip install imglyb-examples
    https://github.com/hanslovsky/imglyb-learnathon
\end{frame}

\section{Examples}
\begin{frame}
    \frametitle{Examples}
    
\end{frame}

\begin{frame}
    \frametitle{Known Issues}
    \begin{itemize}
          \item Java awt requires wrapper script on OSX
          \item Dask Arrays cannot be wrapped into ImgLib2 cell images
    \end{itemize}
\end{frame}




\end{document}
